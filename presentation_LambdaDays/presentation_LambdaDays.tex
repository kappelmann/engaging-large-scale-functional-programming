% \documentclass{beamer}
\documentclass[aspectratio=169]{beamer}
\usepackage{appendixnumberbeamer}

\mode<presentation>{\usetheme[subsectionpage=progressbar,block=fill,numbering=none]{metropolis}}

\usepackage[sfdefault]{FiraSans} %% option 'sfdefault' activates Fira Sans as the default text font
\usepackage[english]{babel}
\usepackage[utf8]{inputenc}

\usepackage{xcolor}
\usepackage{float}
\usepackage{subcaption}
\usepackage{graphicx} % Allows including images
\usepackage[export]{adjustbox}
\usepackage{pgfplots}
\usepackage{caption}
\usepackage{booktabs} % Allows the use of \toprule, \midrule and \bottomrule in tables
\usepackage{multicol}

% Math packages
\usepackage{amsmath}
\usepackage{mathtools}
\usepackage{amssymb}
\usepackage{mathpartir}
\usepackage{stmaryrd}
\usepackage{centernot}
\usepackage[normalem]{ulem}
\usepackage[absolute,overlay]{textpos}

% Coloured boxes
\usepackage{tcolorbox}
\colorlet{alert}{mLightBrown}
\colorlet{mLightBrownTransparent}{white!70!mLightBrown}
\newtcolorbox{alertbox}
{standard jigsaw, opacityback=0,colframe=alert}
\newtcolorbox{tbox}
{standard jigsaw, opacityback=0,opacityframe=0}

%% Coding
\usepackage[outputdir=build]{minted}
\setminted{highlightcolor=mLightBrown!70}
\usepackage{listings}
\lstdefinelanguage{cyp}{
  keywords={Lemma,goal,Proof,QED,To,show,Case,IH},
  keywordstyle=\color{blue}\bfseries,
  keywords=[2]{data},
  keywordstyle=[2]{\color{purple}\bfseries},
  keywords=[3]{.=.},
  keywordstyle=[3]{\color{red}\bfseries},
  comment=[l]{--},
  commentstyle=\color{purple}\ttfamily,
}

\lstset{
  language=cyp,
  extendedchars=true,
  basicstyle=\ttfamily,
  showstringspaces=false,
  showspaces=false,
  tabsize=2,
  breaklines=true,
  showtabs=false
}

% \newcommand{\haskell}[1]{\mintinline{haskell}{#1}}
\newcommand{\backgroundimage}{
% \adjincludegraphics[height=\paperheight,trim={{.15\width} 0 {.15\width} 0},clip]{assets/krakowbridge}
\adjincludegraphics[height=\paperheight]{assets/krakowbridge}
}

\title{Engaging, Large-Scale Functional Programming Education in Physical and Virtual Space} % the title on the title page

\author{\underline{Kevin Kappelmann}, Jonas Rädle, \underline{Lukas Stevens}} % Your name
\institute[TU Munich]{Technical University of Munich}
\date{July 28, 2022} % Date, can be changed to a custom date

\begin{document}

{\usebackgroundtemplate{\backgroundimage}
\begin{frame}[plain]
\titlepage

\tikz[remember picture, overlay]
{\node[anchor=south east, outer sep=0.5cm] at (current page.south east)
{\includegraphics[width=0.2\textwidth]{assets/lambdadays}};}
\end{frame}
}

%------------------------------------------------
\section{Challenges}
% Soaring Enrolments
\begin{frame}{Soaring Enrolments}
\only<1>{\centerline{\Large{\alert{1. Number of Computer Science students exploded}}}}
\only<2-3>{\centerline{Example: Computer Science at TU Munich}

\pause
\begin{multicols}{2}
\begin{figure}
\centering
\captionsetup{justification=centering}
% somehow this needs to be 1.02 when it is 1 below
\resizebox{0.75\linewidth}{!}{
\begin{tikzpicture}
\begin{axis}[
    ymin=0,
    x tick label style={/pgf/number format/.cd,%
      set thousands separator={}}
  ]
  \addplot +[color=red, mark options={fill=red}] coordinates {
    (2013, 3555)
    (2014, 3815)
    (2015, 4240)
    (2016, 4744)
    (2017, 5399)
    (2018, 5986)
    (2019, 6458)
    (2020, 7444)
    (2021, 8253)
  };
\end{axis}
\end{tikzpicture}
}
\caption*{Number of CS students\\(132\% increase)}
\end{figure}

\columnbreak

\begin{figure}
\centering
\captionsetup{justification=centering}
\resizebox{0.75\linewidth}{!}{
\begin{tikzpicture}
\begin{axis}[
    ymin=0,
    x tick label style={/pgf/number format/.cd,%
      set thousands separator={}}
  ]
  \addplot coordinates {
    (2013, 439)
    (2014, 449)
    (2015, 432)
    (2016, 436)
    (2017, 438)
    (2018, 450)
    (2019, 500)
    (2020, 529)
    (2021, 573)
  };
\end{axis}
\end{tikzpicture}
}
\caption*{Number of CS academic staff\\(31\% increase)}
\end{figure}
\end{multicols}
}
\only<1-3>{
\visible<3>{\centerline{\Large{\alert{1000+ students per course are the new normal}}}}
}
% \only<4>{
% Faced by
% \begin{itemize}
  % \item 2019: 13 student assistants
  % \item 2020: 22 student assistants
% \end{itemize}
% }
\end{frame}

\begin{frame}{The Pandemic}
\only<1>{\centerline{\Large{\alert{2. Radical transition to online classes}}}}
\only<2>{\centerline{How can we go from here\dots}
\begin{figure}
\includegraphics[height=0.80\textheight]{assets/class}
\end{figure}
}
\only<3>{\centerline{to here\dots}
\begin{figure}
\includegraphics[height=0.80\textheight]{assets/zoom}
\end{figure}
}
\only<4>{\centerline{without ending up here?}
\begin{figure}
\includegraphics[height=0.80\textheight]{assets/zoom_fatigue}
\end{figure}
}
\end{frame}

\begin{frame}{Usefulness of Functional Programming}
\only<1>{
\centering\Large{\alert{3. Students question the usefulness of functional languages beyond academia}}}
\only<2>{
\begin{multicols}{2}

\begin{figure}
\includegraphics[height=0.80\textheight]{assets/xkcd_haskell}
\caption*{\url{xkcd.com/1312}}
\end{figure}

\columnbreak

\begin{figure}
\includegraphics[height=0.80\textheight]{assets/xkcd_tailrec}
\caption*{\url{xkcd.com/1270}}
\end{figure}
\end{multicols}
}
\end{frame}

\begin{frame}
\centerline{\alert{\huge{There is hope!}}}
\begin{itemize}[<+->]
\item We managed to cope with all these challenges
\item We share our insights, tools, and exercises for other educators
\end{itemize}

\visible<2->{
  You can find our resources on:\\
  \centerline{\small\url{github.com/kappelmann/engaging-large-scale-functional-programming}}
}

\vspace{\baselineskip}
\visible<3->{
Note: We used Haskell, but most ideas apply to any functional programming course
}
\end{frame}

%------------------------------------------------
% \section{Course Structure and Conditions}
% \begin{frame}{Hard Facts}
% \begin{itemize}[<+->]
  % \item Typical functional programming introductory course\\
    % Special: Formal proofs and proof checker
  % % \item 5 ECTS course (${\sim}150$ hours of work)
  % \item 14 weeks; each week: one 90-minute lecture, one 90-minute tutorial, and one homework sheet
  % % \item Students took courses on Java, algorithms and data structures, discrete mathematics, and linear algebra.
   % \item Winter semester 2019: 1057 students and 13 student assistants\\
     % Winter semester 2020 (online): 1031 students and 22 student assistants
% \end{itemize}
% \end{frame}

% \begin{frame}{Syllabus Excerpt}
% We used Haskell, but most ideas apply to other functional languages.
% \pause
% \begin{itemize}[<+->]
  % \item Basic FP concepts: recursion and pattern matching (numbers and lists), polymorphism
  % \item Formal proofs by structural and computation induction
  % \item More FP concepts: higher-order functions, type classes, algebraic datatypes
  % \item Specifics: modules, IO, lazy evaluation, tail recursion
% \end{itemize}
% \end{frame}

%------------------------------------------------
% \section{Lectures}
% \begin{frame}{Time Is Short}
  % \centerline{\alert{\Large{See our paper ;)}}}
% \end{frame}


%------------------------------------------------
\section{Practical Part}
\subsection{Engagement Mechanisms}
% \begin{frame}{Grade Bonus}
% \centerline{\Large{\alert{Incentives do work!}}}
% \pause
% \begin{itemize}[<+->]
% \item Bonus on final exam if certain goals are achieved
% \item Multiple ways to obtain points: homework exercises, participation in workshops, programming contests,\dots
% % \item Was once abolished\dots\\
% % \item[] Result: number of homework submissions severely decreased
% \end{itemize}
% \end{frame}

\begin{frame}[fragile]{Instant Feedback}
\only<1-6>{
\centerline{\Large{\alert{Feedback must come fast!}}}
\pause
\begin{itemize}[<+->]
% \item Asynchronous Q\&A forum
\item Automated testing and feedback
\begin{itemize}
  \item \emph{ArTEMiS} runs tests, manages scores, offers exam mode,\dots
  \item \emph{Tasty} combines \only<4>{\underline}{QuickCheck}, SmallCheck, and HUnit tests
  \item \emph{HLint} for stylistic feedback
  \item \emph{Check Your Proof} for automated proof checking
\end{itemize}
% \pause
% \item Manual reviews turned out to be inefficient\dots
% \begin{itemize}
  % \item \emph{HLint} offers feedback more directly
  % \item Student assistants create engaging exercises instead
% \end{itemize}
\end{itemize}
}

\begin{onlyenv}<7>
\begin{lstlisting}
Lemma: xs ++ (ys ++ zs) .=. (xs ++ ys) ++ zs
Proof by induction on List xs
Case []
  To show: [] ++ (ys ++ zs) .=. ([] ++ ys) ++ zs
  Proof
                    [] ++ (ys ++ zs)
    (by def ++) .=. ys ++ zs
    (by def ++) .=. ([] ++ ys) ++ zs
  QED
Case x : xs
  To show: (x : xs) ++ (ys ++ zs) .=. ((x : xs) ++ ys) ++ zs
  IH: xs ++ (ys ++ zs) .=. (xs ++ ys) ++ zs
  Proof
  ...
\end{lstlisting}
\end{onlyenv}

\end{frame}

\begin{frame}{Workshops With Industry Partners}
\vspace{\baselineskip}

\centerline{\Large{\alert{Functional programming is practical!}}}
\pause
\begin{itemize}
\item In 2020, we hosted 3 workshops with industry partners
\begin{enumerate}[<3->]
  \item Design patterns for functional programs
  \item Networking and advanced IO
  \item User interfaces and functional reactive programming
\end{enumerate}
\item<4-> Great success: 120 registrations for 105 spots
\item<5-> In some cases, workshop extended for multiple hours
\end{itemize}
\visible<6->{
\begin{alertbox}
\centerline{Maybe you want to offer a workshop as well? :)}
\end{alertbox}
}
\end{frame}
% \begin{frame}{Workshops With Industry Partners}
% \vspace{\baselineskip}

% \centerline{\Large{\alert{Functional programming is practical!}}}
% \pause
% \begin{itemize}
% \visible<2->{
% \item In 2020, we hosted three workshops}
% % \begin{enumerate}
  % % \item Design patterns for functional programs
  % % \item Networking and advanced IO
  % % \item User interfaces and functional reactive programming
% % \end{enumerate}
% \only<6->{
% \visible<6->{
% \item Great success: 120 registrations for 105 spots}
% \visible<7->{
% \item In some cases, workshop extended for multiple hours}
% \visible<8->{
% \item Little organisational work}}
% \end{itemize}
% \only<1-5>{\vspace{-\baselineskip}}
% \only<1-3>{
% \visible<3>{
% \centering
% \includegraphics[width=0.9\linewidth]{assets/workshop_slide_active_group.pdf}}}
% \only<4>{
% \centering
% \includegraphics[width=0.9\linewidth]{assets/workshop_slide_networking.pdf}}
% \only<5>{
% \centering
% \includegraphics[width=0.9\linewidth]{assets/workshop_slide_ui.pdf}}
% \end{frame}

% \begin{frame}{Social Interactions}
% \centerline{\Large{\alert{The social environment matters!}}}
% \pause
% \begin{itemize}[<+->]
% % \item Truism: students spend more time on courses they like
% \item Online courses are isolating\dots
% \item[] so let us foster social interaction:
% \begin{itemize}[<4->]
  % \item Pair-programming in tutorials
  % \item ACM-ICPC-like programming contest
  % \item Get-together hangout sessions
  % \item Award ceremonies
% \end{itemize}
% \end{itemize}
% \end{frame}

\begin{frame}[fragile]{Diverse Challenges}
\centerline{\Large{\alert{Offer diverse challenges!}}}
\pause
\begin{itemize}[<+->]
\item Weekly competition exercises
\end{itemize}
\end{frame}

%------------------------------------------------
\begin{frame}[fragile]{Diverse Challenges}

\only<1>{
  % \centerline{Strategy}
\begin{figure}
\includegraphics[height=0.9\textheight]{assets/strategy}
\end{figure}
}

\only<2>{
  % \centerline{Creativity}
\begin{figure}
% \includegraphics[height=0.85\textheight]{assets/creativity}
\includegraphics[height=0.9\textheight]{assets/contest}
\end{figure}
}
\only<3>{
\begin{columns}[t]
\begin{column}{0.35\textwidth}
  % \centerline{Creativity}
\begin{figure}
% \includegraphics[height=0.85\textheight]{assets/creativity}
% \includegraphics[width=\textwidth]{assets/creativity}
  \begin{tikzpicture}[scale=0.65]
  \begin{axis}[
    minor tick num=1,
    samples=120,
    axis y line=left,
    axis x line=middle
    ]
    \addplot+[domain=0:10, mark=none, teal]
      coordinates {(0,0) (2, 1)};
    \addplot+[domain=0:10, mark=none, blue]
      coordinates {(2,1) (3, 0.5)};
    \addplot+[domain=0:10, mark=none, purple]
      coordinates {(3, 0.5) (7.5, 0.5)};
    \addplot+[domain=0:10, mark=none, orange]
      coordinates {(7.5, 0.5) (10, 0)};
    \addplot[densely dashed] coordinates {(2,0) (2,1)};
    \addplot[densely dashed] coordinates {(3,0) (3,0.5)};
    \addplot[densely dashed] coordinates {(7.5,0) (7.5,0.6)};
    \addplot[densely dashed] coordinates {(10,0) (10,0.6)};
    \addplot[->] coordinates {(0,0.9) (2,0.9)} node[midway, above, teal] {attack};
    \addplot[->] coordinates {(2,0.4) (3,0.4)} node[midway, below, blue] {decay};
    \addplot[->] coordinates {(6,0) (6,0.5)} node[midway,left, purple] {sustain};
    \addplot[<-] coordinates {(7.5,0.6) (10,0.6)} node[midway, below, orange] {release};
    \addplot[->] coordinates {(0,0.1) (10,0.1)} node[midway, above, olive] {duration};
  \end{axis}
  \end{tikzpicture}

\end{figure}
\end{column}
\begin{column}{0.65\textwidth}
% }
% \only<4>{
  % \centerline{Creativity}
\begin{figure}
\includegraphics[width=\textwidth]{assets/creativity2}
\end{figure}
\end{column}
\end{columns}
}
\end{frame}

\begin{frame}[fragile]{Diverse Challenges}
\centerline{\Large{\alert{Offer diverse challenges!}}}
\begin{itemize}
\item Weekly competition exercises
% \begin{itemize}[<3->]
  % \item Code-golfing
  % \item Optimisations (SAT and resolution provers, mathematical computations,\dots)
  % \item Games strategies and tournaments
  % \item Creative tasks: art and music generation
% \end{itemize}
% \item Accompanied by humorously written blog posts
\item Works extremely well to motivate talented students.
\item<2-> Awards for top 30 students
% \item[]<3-> \emph{but it is very time-consuming}.
\end{itemize}
% \begin{onlyenv}<2>
% \begin{minted}[fontsize=\small]{haskell}
% bernoulli = genericIndex $ map head $ iterate
  % ((*) `zipWith` enumFrom 1 <<< zipWith subtract =<< tail)
  % recip `map` enumFrom 1
% \end{minted}
% \end{onlyenv}
\visible<3->{
\begin{alertbox}
\centerline{Maybe you want to offer awards or challenges as well? :)}
\end{alertbox}
}
\end{frame}


%------------------------------------------------
\section{I/O Mocking}

\begin{frame}{Motivation}

\begin{itemize}[<+->]
  \item Submissions (primarily) tested with QuickCheck
  \item I/O is an important part of the syllabus
\end{itemize}
\vfill
\visible<3->{
\begin{center}
\Large \alert{So how do we test I/O in Haskell?}
}
\end{center}

\end{frame}

\begin{frame}[fragile]{The Standard Way}
\begin{visibleenv}<3->
\begin{minted}[fontsize=\small]{haskell}
import qualified Prelude
import Prelude hiding (readFile, writeFile)

class Monad m => MonadFileSystem m where
  readFile :: FilePath -> m String
  writeFile :: FilePath -> String -> m ()
\end{minted}
\end{visibleenv}

\vspace{1.5em}

\begin{onlyenv}<1>
\begin{minted}[fontsize=\small]{haskell}
copyFile :: FilePath -> FilePath -> IO ()
copyFile = _
\end{minted}
\end{onlyenv}

\begin{onlyenv}<2-3>
\begin{minted}[fontsize=\small]{haskell}
copyFile :: MonadFileSystem m =>
            FilePath -> FilePath -> m ()
copyFile = _
\end{minted}
\end{onlyenv}

\begin{visibleenv}<4->
\begin{minted}[fontsize=\small]{haskell}
copyFile :: MonadFileSystem m =>
            FilePath -> FilePath -> m ()
copyFile source target = do
  content <- readFile source
  writeFile target content
\end{minted}
\end{visibleenv}
\end{frame}

\begin{frame}[fragile]{Multiple Instantiations}
\begin{minted}[fontsize=\small]{haskell}
instance MonadFileSystem IO where
  readFile = Prelude.readFile
  writeFile = Prelude.readFile
\end{minted}
\begin{visibleenv}<2>
\begin{minted}[fontsize=\small]{haskell}

data MockFileSystem =
  MockFileSystem (Map FilePath String)
instance MonadFileSystem (State MockFileSystem) where
  readFile = _
  writeFile = _
\end{minted}
\end{visibleenv}
\end{frame}

\begin{frame}[fragile]{The Problem}
What is the problem with
\begin{minted}[fontsize=\small]{haskell}
copyFile :: MonadFileSystem m =>
            FilePath -> FilePath -> m ()
copyFile = _
\end{minted}
\pause
\vfill
\begin{center}
  \alert{\Large Lack of transparency!}
\end{center}
\end{frame}

\begin{frame}{The Solution}
  \large
  Delay mocking to the compliation stage\\[1.5em]
  \pause
  \qquad by replacing the \texttt{IO} module with a mixin.
\end{frame}

\begin{frame}[fragile]{The Mixin}
\begin{minted}[fontsize=\small]{haskell}
data RealWord = RealWord {
  workDir :: FilePath,
  files :: Map File Text,
  handles :: Map Handle HandleData,
  user :: IO (),
  ...
}
\end{minted}
\begin{visibleenv}<2->
\begin{minted}[fontsize=\small]{haskell}

newtype IO a = IO { unwrapIO ::
  ExceptT IOException (PauseT (State RealWorld)) a }
\end{minted}
\end{visibleenv}
\end{frame}

\begin{frame}[fragile]{The \texttt{Pause} Monad}
\begin{minted}[fontsize=\small]{haskell}
class Monad m => MonadPause m where
  pause :: m ()
  stepPauseT :: m a -> m (Either (m a) a)
\end{minted}
\end{frame}

\begin{frame}[fragile]{An Example Interaction}
  \begin{columns}[t]
    \begin{column}{0.5\textwidth}
\begin{onlyenv}<1>
\begin{minted}[fontsize=\small, frame=topline, framesep=10pt, label={Student submission}, labelposition=topline, highlightlines={1}]{haskell}
main = do
  x <- getLine
  putStrLn $ "Hi " ++ x
\end{minted}
\end{onlyenv}
\begin{onlyenv}<2>
\begin{minted}[fontsize=\small, frame=topline, framesep=10pt, label={Student submission}, labelposition=topline, highlightlines={2}]{haskell}
main = do
  x <- getLine
  putStrLn $ "Hi " ++ x
\end{minted}
\end{onlyenv}
\begin{onlyenv}<3-5>
\begin{minted}[fontsize=\small, frame=topline, framesep=10pt, label={Student submission}, labelposition=topline, highlightlines={2}, highlightcolor=lightgray]{haskell}
main = do
  x <- getLine
  putStrLn $ "Hi " ++ x
\end{minted}
\end{onlyenv}
\begin{onlyenv}<6>
\begin{minted}[fontsize=\small, frame=topline, framesep=10pt, label={Student submission}, labelposition=topline, highlightlines={2}]{haskell}
main = do
  x <- getLine
  putStrLn $ "Hi " ++ x
\end{minted}
\end{onlyenv}
\begin{onlyenv}<7>
\begin{minted}[fontsize=\small, frame=topline, framesep=10pt, label={Student submission}, labelposition=topline, highlightlines={3}]{haskell}
main = do
  x <- getLine
  putStrLn $ "Hi " ++ x
\end{minted}
\end{onlyenv}
\begin{onlyenv}<8->
\begin{minted}[fontsize=\small, frame=topline, framesep=10pt, label={Student submission}, labelposition=topline]{haskell}
main = do
  x <- getLine
  putStrLn $ "Hi " ++ x
\end{minted}
\end{onlyenv}
    \end{column}
%-----------------------------------
    \begin{column}{0.5\textwidth}
\begin{onlyenv}<1-2>
\begin{minted}[fontsize=\small, frame=topline, framesep=10pt, label={Mock user}, labelposition=topline, highlightlines={1},highlightcolor=lightgray]{haskell}
user s = do
  hPutStrLn stdin s
  out <- hGetLine stdout
  when (out /= _)
       (fail $ _)
\end{minted}
\end{onlyenv}
\begin{onlyenv}<3>
\begin{minted}[fontsize=\small, frame=topline, framesep=10pt, label={Mock user}, labelposition=topline, highlightlines={1}]{haskell}
user s = do
  hPutStrLn stdin s
  out <- hGetLine stdout
  when (out /= _)
       (fail $ _)
\end{minted}
\end{onlyenv}
\begin{onlyenv}<4>
\begin{minted}[fontsize=\small, frame=topline, framesep=10pt, label={Mock user}, labelposition=topline, highlightlines={2}]{haskell}
user s = do
  hPutStrLn stdin s
  out <- hGetLine stdout
  when (out /= _)
       (fail $ _)
\end{minted}
\end{onlyenv}
\begin{onlyenv}<5>
\begin{minted}[fontsize=\small, frame=topline, framesep=10pt, label={Mock user}, labelposition=topline, highlightlines={3}]{haskell}
user s = do
  hPutStrLn stdin s
  out <- hGetLine stdout
  when (out /= _)
       (fail $ _)
\end{minted}
\end{onlyenv}
\begin{onlyenv}<6-7>
\begin{minted}[fontsize=\small, frame=topline, framesep=10pt, label={Mock user}, labelposition=topline, highlightlines={3}, highlightcolor=lightgray]{haskell}
user s = do
  hPutStrLn stdin s
  out <- hGetLine stdout
  when (out /= _)
       (fail $ _)
\end{minted}
\end{onlyenv}
\begin{onlyenv}<8>
\begin{minted}[fontsize=\small, frame=topline, framesep=10pt, label={Mock user}, labelposition=topline, highlightlines={3}]{haskell}
user s = do
  hPutStrLn stdin s
  out <- hGetLine stdout
  when (out /= _)
       (fail $ _)
\end{minted}
\end{onlyenv}
\begin{onlyenv}<9>
\begin{minted}[fontsize=\small, frame=topline, framesep=10pt, label={Mock user}, labelposition=topline, highlightlines={4,5}]{haskell}
user s = do
  hPutStrLn stdin s
  out <- hGetLine stdout
  when (out /= _)
       (fail $ _)
\end{minted}
\end{onlyenv}
\begin{onlyenv}<10>
\begin{minted}[fontsize=\small, frame=topline, framesep=10pt, label={Mock user}, labelposition=topline]{haskell}
user s = do
  hPutStrLn stdin s
  out <- hGetLine stdout
  when (out /= _)
       (fail $ _)
\end{minted}
\end{onlyenv}
    \end{column}
  \end{columns}
\end{frame}

%------------------------------------------------
% \section{Check Your Proof}

% \begin{frame}{CYP In A Nutshell}

% \begin{itemize}[<+->]
  % \item Operates on a strict, untyped subset of Haskell
  % \item Automatically checks simple inductive proofs
  % % \item Provides instant feedback
  % \item Integrates with Tasty
% \end{itemize}
% \end{frame}

% \begin{frame}[fragile]{Background Theory}
% \lstset{language=CYP}
  % \begin{lstlisting}
% data List a = [] | a : List a

% [] ++ ys = ys
% (x : xs) ++ ys = x : (xs ++ ys)

% goal xs ++ (ys ++ zs) .=. (xs ++ ys) ++ zs
  % \end{lstlisting}
% \end{frame}

% \begin{frame}[fragile]{The \texttt{[]} Case}
  % \begin{lstlisting}
% Lemma: xs ++ (ys ++ zs) .=. (xs ++ ys) ++ zs
% Proof by induction on List xs
  % Case []
    % To show: [] ++ (ys ++ zs) .=. ([] ++ ys) ++ zs
    % Proof
                      % [] ++ (ys ++ zs)
      % (by def ++) .=. ys ++ zs
      % (by def ++) .=. ([] ++ ys) ++ zs
    % QED
  % \end{lstlisting}
% \end{frame}

% \begin{frame}[fragile]{The \texttt{Cons} Case}
  % \begin{lstlisting}
  % Case x : xs
    % To show: (x : xs) ++ (ys ++ zs)
         % .=. ((x : xs) ++ ys) ++ zs
    % IH: xs ++ (ys ++ zs) .=. (xs ++ ys) ++ zs
    % Proof
                        % (x : xs) ++ (ys ++ zs)
      % (by def ++)   .=. x : (xs ++ (ys ++ zs))
      % (by IH)       .=. x : ((xs ++ ys) ++ zs)

                        % ((x : xs) ++ ys) ++ zs
      % (by def ++)   .=. (x : (xs ++ ys)) ++ zs
      % (by def ++)   .=. x : ((xs ++ ys) ++ zs)
    % QED
% QED
  % \end{lstlisting}
% \end{frame}

% \begin{frame}{Our Experience With CYP}
% \begin{itemize}[<+->]
  % \item Student feedback 18 positive, 3 negative
  % \item Main criticism: lack of documentation
  % \item Mostly well-structured inductive proofs in the exam
% \end{itemize}
% \end{frame}

\begin{frame}
\centerline{\alert{\huge{Find more in our repository!}}}
\begin{itemize}
% \item I/O mocking framework
\item Games, music synthesiser, turtle graphics,\dots
\item Proof checker for inductive and equational reasoning
\item More engagement mechanisms and insights, our technical setup,\dots
\end{itemize}

\vspace{\baselineskip}
\centerline{\small\url{github.com/kappelmann/engaging-large-scale-functional-programming}}
\end{frame}

%------------------------------------------------
% \section{Future Work}

% \begin{frame}{Future Work}
% \centerline{Preventing collaboration/cheating}
% \begin{figure}
% \includegraphics[height=0.75\textheight]{assets/group}
% \end{figure}
% \end{frame}

%------------------------------------------------
{\usebackgroundtemplate{\backgroundimage}
\begin{frame}
\begin{center}
\Huge{\textbf{\alert{Any questions?}}}

\vfill
\small{
Thanks to Tobias Nipkow, Manuel Eberl, our student assistants,
our industry partners (Active Group, QAware, TNG Technology Consulting, and Well-Typed),
and\\ our 2000 Haskell students}
\end{center}
\end{frame}}
%----------------------------------------------------------------------------------------
%\begin{frame}[allowframebreaks]{References}
  %\bibliography{../paper/sources.bib}
  %\bibliographystyle{abbrv}
%\end{frame}

% \begin{frame}[allowframebreaks]{Image Sources}
% \begin{itemize}
% \end{itemize}
% \end{frame}

\end{document}
