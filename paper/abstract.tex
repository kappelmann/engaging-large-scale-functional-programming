Worldwide, computer science departments have experienced a dramatic increase in the number of student enrolments.
Moreover, the ongoing COVID-19 pandemic requires institutions to radically replace the traditional way of on-site teaching,
moving interaction from physical to virtual space.
We report on our strategies and experience tackling these issues
as part of a Haskell-based functional programming and verification course,
accommodating over 2000 students in the course of two semesters.
Among other things,
we fostered engagement with weekly programming competitions
and creative homework projects,
workshops with industry partners,
and collaborative pair-programming tutorials.
To offer such an extensive programme to hundreds of students,
we automated feedback for programming as well as
inductive proof exercises.
We explain and share our tools and exercises so that they can be reused by other educators.

% Notes: what should be in the abstract
% \begin{enumerate}
  % \item rising number of students
  % \item online teaching
  % \item keep quality high
  % \item give good immediate feedback
  % \item keep engagement and interaction high
  % \item do not overwhelm; step by step introduction
  % \item give incentives
  % \item offer space for creativity
% \end{enumerate}
