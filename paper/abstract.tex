World-wide, computer science departments have experienced a dramatic increase in the number of student enrollments.
Moreover, the ongoing COVID-19 pandemic requires institutions to radically replace the traditional way of on-site teaching,
moving interaction from physical to virtual space.
We report on our strategies and experience tackling these issues
as part of a Haskell-based functional programming and verification course,
accommodating \kevin{TODO: find a better word} over TODO students in the course of two semesters.
Among other things,
we fostered engagement with weekly programming competitions
and creative homework projects,
special workshops with industry partners,
and collaborative pair-programming tutorials.
To offer such an extensive programme to hundres of students,
we automated all feedback for programming as well
inductive proof exercises.

% Notes: what should be in the abstract
% \begin{enumerate}
  % \item rising number of students
  % \item online teaching
  % \item keep quality high
  % \item give good immediate feedback
  % \item keep engagement and interaction high
  % \item do not overwhelm; step by step introduction
  % \item give incentives
  % \item offer space for creativity
% \end{enumerate}
