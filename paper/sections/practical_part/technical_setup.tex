\subsection{Technical Setup and Automated Assessment}\label{sec:tech_setup_test}

\paragraph{Automated Assessment}
In WS19, we used an improved version of
the testing infrastructure introduced in~\cite{next_1100}.
However, this system was not able to manage online exams or mark non-programming tasks.
We thus switched to a newly written open-source
tool developed at TUM called ArTEMiS~\cite{artemis}.
ArTEMis is a highly scalable, automated assessment management system and is programming language independent --
it only expects test runners to produce tests results
adhering to the Apache Ant JUnit XML schema.
It already offered support for a few imperative programming languages,
and we added support for Haskell\footnote{It now also supports OCaml}.

As ArTEMis takes care of most things,
including load-balancing and automated test execution,
and offers an exam mode and good support for grading non-programming tasks,
the only thing we were left to do was writing the actual test code.
For the most part, we verified the results computed
by a student's submission by comparing them to those
computed by a sample solution written by us.
In some cases, we also tested for efficiency using timeouts.
Our tests were powered by the following libraries:
\begin{enumerate}
  \item QuickCheck \cite{quickcheck}:
  Although QuickCheck provides means to automatically generate test data (using the typeclass \lstinline!Arbitrary!),
  most tests and benchmarks used customly written QuickCheck input generators.
  This was necessery to increase coverage and eliminate non-applicable inputs for tests with preconditions.
  We also used custom shrinkers to provide better feedback to students in case of a failure.
  In both cases, the flexible combinators provided by QuickCheck made this a straightforward task.
  \item SmallCheck \cite{smallcheck}: The exhaustive testing facilities provided by SmallCheck mainly served
    as a complementary tool that provided small counterexamples for, in many cases, obvious deficiencies.
  \item Tasty\footnote{\url{https://hackage.haskell.org/package/tasty}}: We used Tasty to put QuickCheck, SmallCheck, and unit tests as well as the checking of ``Check Your Proof'' proofs (see \cref{sec:cyp}) into one common framework that is capable to generate results interpretable by ArTEMiS.
  We used the unit testing facilities by Tasty to complement our QuickCheck/SmallCheck tests with corner cases.
  Integration of proof checking was pleasently straightforward.
  Moreover, Tasty supports timeouts for individual test cases,
  solving the issue of truncated test reports mentioned in~\cite{next_1100}.
\end{enumerate}

\paragraph{Online Tutorials}
In previous iterations,
there were no recommendations to the students about the programming environment they should use for the practical part of the course.
However, due to the COVID-19 pandemic,
this was no longer an option:
we needed a way for students to share their code in read and write mode
to other peers (pair-programming) and their teaching assistant (for feedback purposes) during online tutorials.
Moreover, as explained in \cref{sec:engagement},
we decided to give up on manual feedback on code quality
and wanted students to use a linter instead.
Finally, we had negative experiences with students
installing no compiler at all
and instead (mis)using our submission server as a compilation backend.
We thus introduced a strict policy
for the technical setup to be used during
the tutorials.
A general overview can be found online\footnote{\url{https://www21.in.tum.de/teaching/fpv/WS20/installation.html}},
and an extended description and report will follow in the final version of this article.

In short, we can report positively on the setup.
We have experience with running online tutorials for other large lectures and can testify that
the system in this paper worked best in technical as well as social aspects.
We can also report that students were more knowledgeable
about their Haskell programming environment (compilation, dependency management, etc.) than in previous iterations.
Misappropriation of the submission server as a compiler backend also stopped.
