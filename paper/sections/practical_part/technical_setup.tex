\subsection{Technical Setup and Automated Assessment}\label{sec:tech_setup_test}

\paragraph{Automated Assessment}
In WS19, we used an improved version of
the testing infrastructure introduced in \citep{next_1100}.
However, this system was not able to manage online exams and mark freeform submissions.
We thus switched to use a newly written open-source
tool developed at TUM called ArTEMiS \citep{artemis}.
ArTEMis is a highly scalable, automated assessment management system.
It already offered support for a few imperative programming languages
and we added support for Haskell\footnote{It now also supports OCaml}.
As ArTEMis takes care of anything from load-balancing to
automated testing and
offers a special exam mode and good support for freeform submissions,
the only thing we were left to do is writing the actual test code.
More details about ArTEMis and our test setup follow
in the extended version of this article.

\paragraph{Online Tutorials}
In previous iterations,
no recommendations about the editor to be used for the practical part of the course
were given.
However, due to the COVID19 pandemic,
this was no longer an option:
we needed a way for students to share their code in read and write mode
to other peers (pair-programming) and their tutor (for feedback purposes).
Moreover, as explained in \cref{sec:engagement},
we decided to give up on manual feedback on code quality
and let students use a linter instead.
Finally, we had very negative experiences with students
installing no compiler at all
and instead (mis)using our submission server as a compilation backend.
We thus introduced a strict policy
for the technical setup that has to be used as part of
the tutorials.
A general overview can be found online\footnote{\url{https://www21.in.tum.de/teaching/fpv/WS20/installation.html}},
and an extended description and report will follow in the final version of the article.

In short, we can report very positvely on the setup.
We have experience with running online tutorials for other large lectures and can testify that
the system in this paper worked best in technical as well as social aspects.
We can also report that students were more knowledgable
about the Haskell editor toolchain than in previous iterations.
Abuses
Misappropriation of the submission server as a compiler backend also stopped.

