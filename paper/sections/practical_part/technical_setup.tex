\subsection{Technical Setup and Automated Assessment}\label{sec:tech_setup_test}

\paragraph{Automated Assessment}
In WS19, we used an improved version of
the testing infrastructure introduced in \cite{next_1100}.
However, this system was not able to manage online exams or mark non-programming tasks.
We thus switched to a newly written open-source
tool developed at TUM called ArTEMiS \cite{artemis}.
ArTEMis is a highly scalable, automated assessment management system.
It already offered support for a few imperative programming languages
and we added support for Haskell\footnote{It now also supports OCaml}.
As ArTEMis takes care of most things,
including load-balancing and automated test execution,
and offers an exam mode and good support for grading non-programming tasks,
the only thing we were left to do was writing the actual test code.

More details about ArTEMis and our test setup will follow
in the final version of this article.
\kevin{TODO: mention QuickCheck, tasty, etc. and remove ``more details''}

\paragraph{Online Tutorials}
In previous iterations,
there were no recommendations to the students about the programming environment they should use for the practical part of the course.
However, due to the COVID-19 pandemic,
this was no longer an option:
we needed a way for students to share their code in read and write mode
to other peers (pair-programming) and their teaching assistant (for feedback purposes) during online tutorials.
Moreover, as explained in \cref{sec:engagement},
we decided to give up on manual feedback on code quality
and wanted students to use a linter instead.
Finally, we had negative experiences with students
installing no compiler at all
and instead (mis)using our submission server as a compilation backend.
We thus introduced a strict policy
for the technical setup to be used during
the tutorials.
A general overview can be found online\footnote{\url{https://www21.in.tum.de/teaching/fpv/WS20/installation.html}},
and an extended description and report will follow in the final version of this article.

In short, we can report positively on the setup.
We have experience with running online tutorials for other large lectures and can testify that
the system in this paper worked best in technical as well as social aspects.
We can also report that students were more knowledgeable
about their Haskell programming environment (compilation, dependency management, etc.) than in previous iterations.
Misappropriation of the submission server as a compiler backend also stopped.
