\subsection{Engagement Mechanisms}\label{sec:engagement}

Students spend the majority of their time on the practical part of the course.
This is where they apply the theory explained in the lecture in tutorial and homework exercises in the form of programming tasks, proof exercises, and miscellaneous assignments (type inference, transformation of programs into tail-recursive form, etc.).
As each student has unique interests, strengths and weaknesses, and different levels of commitment,
we aim to employ a diverse set of mechanisms to keep them engaged.

As outlined in the introduction, keeping up student engagement is particularly challenging in courses that are taught remotely. We experienced this first hand when we taught a course in theoretical computer science during the first semester affected by the COVID-19 pandemic. We saw a significantly larger decrease in homework and tutorial participation over the course of that semester than in previous iterations. We thus attempted to put a particular emphasis on engaging teaching methods for the functional programming course in WS20.

We want to emphasise that engagement does not simply
increase by offering more things -- this may even increase stress --
but by offering things that serve neglected needs.
Good engagement mechanisms not only keep students busy
but genuinely make the course more fun.
Participation should originate from curiosity and enjoyment rather than pressure.

Here we describe the engagement mechanisms that were particularly valuable for our course:

\paragraph{Grade Bonus}
The most straightforward of all mechanisms is
that of the grade bonus.
For both iterations,
students were able to obtain a bonus of one grade step on their final exam provided that they achieved certain goals in the practical part of the course.
This mechanism was already used in a previous iteration
but subsequently dropped because of negative experiences with plagiarism.
However, as a result, participation in homework exercises
severely decreased \cite{next_1100}.
Moreover, the student council reported to us that one of the most asked for
wishes by students is that of a grade bonus.

We hence re-introduced the bonus with some changes.
First, instead of asking for $40\%$ of all achievable points,
we changed to a pass-or-fail per exercise sheet system.
Students passed a sheet if they
passed $\approx 70\%$ of all tests
and obtained the bonus if they passed $\approx 70\%$ of all sheets.
We changed to this system so that students
could not obtain the grade bonus early on in the semester and then stop participating,
leading to cramming.

Secondly, in WS20, we introduced additional ways to
obtain bonus points,
for example participation in the programming contest or workshops by industry partners.
This diversified the system and offered new ways
to obtain the bonus,
in particular for those students that were struggling
with programming tasks but were nevertheless interested in the course material.

We can report that participation recovered
back to the levels before the grade bonus was dropped in a previous iteration.
In contrast to previous years,
we have not seen any severe cases of plagiarism despite
running all submissions through a plagiarism checking tool\footnote{We used Moss \url{https://theory.stanford.edu/~aiken/moss/}}.
\jonas{TODO: statistics}

\paragraph{Instant Feedback}
An observation we already made in \cref{sec:lectures}
extends to the practical part of the course:
feedback must come fast.
Again, an asynchronous Q\&A forum helps in this regard,
at least for questions of a general nature.
Questions and problems specific to the submission of a student (e.g.\ a bug or error in a proof),
however, must be fixed by the student themselves as 1) it is a critical skill of any computer scientists to discover bugs and 2) code/proofs may not be shared before the submission deadline due to the grade bonus.

Automated tests can fill this gap:
they provide direct feedback (e.g.\ failing input and expected output pairs) with little delay
without giving away too much information.
Needless to say, they are also an important mechanism
to scaling the homework system to a large number of students.
We describe our testing infrastructure in more detail in \cref{sec:tech_setup_test}.

However, we still let student assistants manually review all final submissions
in the first iteration of the course.
We specifically instructed them to provide feedback not covered by automation,
in particular regarding code quality.
To our dismay, we have to report that this feedback did very little and
most of it was probably ignored.
In part, this is because it took 1--2 weeks after each submission deadline to provide feedback to all students.
At that point, the students had already moved on to a fresh set of exercises and were probably not motivated to revisit their old submissions.
Some students may also only care about passing the tests
and are thus not particularly interested in feedback about code quality.

In our second iteration, we hence reallocated resources:
instead of grading submissions,
student assistants now supported us by creating engaging exercises
(see \cref{sec:selected_exercises})
and offering new content (e.g.\ running workshops with industry partners)
while we focused on writing exhaustive tests with good feedback and extended our automated proof checking facilities (see \cref{sec:cyp}).
To provide at least some feedback on
code quality, we instructed students
to use a linter (see \cref{sec:tech_setup_test}).

We can report very positively on this reallocation:
we were able to offer a more diverse set of exercises and
had the resources to offer new content
while quality of code did not seem to suffer.
Indeed, the linter even seemed to increase students' awareness
to not only write correct code but also use good coding patterns.
This seems to be due to the fact that 1) the linter provided instant feedback and 2) it highlighted affected code fragments directly.

\paragraph{Competition and Awards}
Due to positive feedback,
we continued the tradition of running an opt-in weekly
programming competition as introduced in \cite{next_1100}.
Each week, one of the homework assignments
was selected as a competition problem
and a criterion for ranking the correct submissions was fixed.
The range of problems and ranking criteria were diverse,
including code golf challenges,
optimisation problems,
game strategy competitions,
an ACM-like programming contest,
and creative tasks like music composition
and computer art.
We introduce some of the exercises and share their templates in \cref{sec:selected_exercises}.
The top 30 entries received points
and were presented to the public on a blog\footnote{\url{https://www21.in.tum.de/teaching/fpv/WS20/wettbewerb.html} (WS20) and

\url{https://www21.in.tum.de/teaching/fpv/WS19/wettbewerb.html} (WS19)}
written using the ironic self-important third-person style established in previous semesters.
The top 30 students then received awards at
a humorously organised award ceremony at the end of the semester.

To improve on previous iterations,
we made participation more exciting by organising better awards.
We cooperated with industry partners
to offer prizes such as tickets to functional programming conferences,
Haskell workshops and programming books, as well as cash and material prizes.
This initial contact with industry partners
also sparked the idea to offer real-world Haskell workshops
run by software engineers from industry in WS20 (explained further below).

The competition in WS20 also greatly benefited from incorporating the work of our student assistants:
At the beginning of the semester,
we brainstormed for competition ideas.
We then formed teams, each one being responsible for the
implementation of one idea to be published as a competition exercise during the semester.
This allowed us to create more extensive, diverse,
and practical exercises than in previous iterations.

As reported in \cite{next_1100},
we can confirm that the competition works extremely well to motivate talented students.
They go well beyond what is taught as part of the course in coming up with competitive solutions.
Many of them continue to become major drivers in
the team of student assistants in follow-up iterations.
Indeed, after offering the competition in WS19,
we received more applications for student assistant positions in WS20 (they more than doubled) than we could hire -- something we had never experienced before.
We also received testimonies from students that even though they did not perform well or did not participate at all in the competition,
they nevertheless enjoyed the humorous blog posts and advanced material discussed on it.

Despite the work of our student assistants,
however,
running the competition remains an enormously labour-intensive task,
in particular the evaluation of submissions and
the writing of the blog post.
We envisage further assistance by student assistants in future iterations in those regards.

\paragraph{Workshops with Industry Partners}
Many students at TUM have questioned the applicability and value of functional programming for real-world applications.
Obviously, there is not much use in us academics telling
them otherwise.
Instead, we had the idea to invite people from industry
to offer functional programming workshops about
practical topics not covered in our course.

In WS20,
we were able to host three workshops on 1) design patterns
2) networking and advanced IO and 3) user interfaces and multithreading.
We limited participation to 35 students for each workshop,
and to our delight, demand exceeded supply (more than 120 students applied).
Both industry partners and workshop participants
reported very positively to us.
In some cases,
workshops were even extended for multiple hours due to the great curiosity by students.
Moreover, organisational overhead was small:
we merely had to communicate the syllabus to our partners and coordinate time and place.
We envisage offering even more workshops in future iterations and highly recommend this mechanism to demonstrate real-world applicability to other educators.

\paragraph{Social Interactions}
Studies confirmed that the COVID-19 pandemic
worsened students' social life,
leading to higher levels of stress, anxiety, loneliness, and depressive symptoms \cite{students_lockdown1}.
This got us thinking about mechanisms we
can employ in WS20 to foster social interaction and exchange between students
-- which play a crucial pedagogical role in general~\cite{impact_social_interaction}.

Firstly, we decided to employ pair-programming (groups of 3--4 students)
in our online tutorials.
The technical setup for this is described in \cref{sec:tech_setup_test}.
This not only made social interactions an integral part of the tutorial,
but also had positive effects on knowledge sharing.
We can report very positively on this policy.

Secondly, we hosted two informal get-together sessions,
one at the beginning and one at the end of the semester.
We started with icebreaker sessions in breakout rooms,
randomly allocating students and at least one student assistant in each group.
Following these sessions,
we opened thematic breakout rooms where students
could freely move and talk about a given topic.
Some students preferred to talk about the course material, others had
light-hearted conversations about university, yet others started to play online games.
All in all, we received very positive feedback for our get-together sessions.

Thirdly, we organised an
ACM-like programming contest
where students could participate in teams,
followed again by a light-hearted get-together session
for participants.
The contest will be described in more detail in \cref{sec:selected_exercises} in the final version of this article.
