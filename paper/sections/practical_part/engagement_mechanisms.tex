\subsection{Engagement Mechanisms}

\kevin{TODO: probably move to intro}
One main concern of educators around the globe is cramming:
the practice of studying extensively just before the exam while
showing little participation during the semester.
Cramming tends to result in poor long-term retention and shallow understanding
of material.
Indeed, the benefit of spacing learning events apart rather than cramming has been demonstrated in hundreds of experiments \citep{cramming1,cramming2}.

To prevent cramming,
we used a variety of engagement mechanisms and incentives as the course progresses.
Good engagement mechanisms not only keep students busy
but genuinely make the course more fun.
Participation should originate from curiosity and enjoyment rather than pressure.
In the best case, students might even study beyond the material presented
if given the possibility.

\paragraph{Grade Bonus}
Grade bonus: re-introduced after low engagement in last iteration

\paragraph{Instand Feedback}
TODO

\paragraph{Wettbewerb and Awards}
TODO

\paragraph{Social Interactions}
Pair-Programming, FPV\&Chill, Contest

\paragraph{Workshops with Industry Partners}
TODO

