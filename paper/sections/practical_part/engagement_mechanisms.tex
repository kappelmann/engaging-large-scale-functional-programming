\subsection{Engagement Mechanisms}\label{sec:engagement}

\jonas{TODO: online semester makes it especially difficult
to ...}

Students spend the majority of their time on the practical part of the course.
This is where they apply the theory explained in the lecture,
living their own success stories and failures,
shaping their own adventure.
As each student has unique interests, strengths and weaknesses, and their own level of commitment,
we also have to employ a diverse set of mechanisms to keep them engaged.

We want to emphasise that engagement does not simply
increase by offering more things -- even worse, it might even increase stress --
but by offering things that serve neglected needs.
Good engagement mechanisms not only keep students busy
but genuinely make the course more fun.
Participation should originate from curiosity and enjoyment rather than pressure.

Here we describe the engagement mechanisms that were particularly valuable for our course:

\paragraph{Grade Bonus}
The most straightforward of all mechanisms is
that of the grade bonus.
For both iterations,
students were able to obtain a \jonas{TODO: grade step} 0.3 bonus on their final exam provided that they achieved certain goals in the practical part of the course.
This mechanism was already used in a previous iteration
but subsequently dropped because of negative experiences with plagiarism.
However, as a result, participation in homework exercises
severely decreased \cite{next_1100}.
Moreover, the student council reported to us that one of the most asked for
wishes by students is that of a grade bonus.

We hence re-introduced the bonus with some changes.
First, instead of asking for $40\%$ of all achievable points,
we changed to a pass-or-fail per exercise sheet system.
Students passed a sheet if they
passed $\approx 70\%$ of all tests
and obtained the bonus if they passed $\approx 70\%$ of all sheets.
We changed to this system so that students
cannot obtain the grade bonus very early on in the semester and stop participating.

Secondly, in WS20, we introduced additional ways to
obtain points that can be used for the bonus,
for example participation in Haskell programming contests or workshops by industry partners.
This diversified the system and offered new ways for students who were struggling
with programming tasks but nevertheless were interested in the course material
to obtain the bonus.

We can report that participation recovered
back to levels before the grade bonus was dropped in a previous iteration.
In contrast to previous years,
we have not seen any severe cases of plagiarism despite
checking all exercises using a plagarism checking tool.
\kevin{TODO: reference for plagiarism checker}
\jonas{TODO: statistics}

\paragraph{Instant Feedback}
An observation we already made in \cref{sec:lectures}
extends to the practical part of the course:
feedback must come fast.
Again, an asynchronous Q\&A forum helps in this regard,
at least for questions of a general nature.
Questions and problems specific to the submission of a student (e.g.\ a bug or error in a proof),
however, must be fixed by the student themselves as 1) it is a critical skill of any computer scientists to disover bugs and 2) code/proofs cannot be shared in public due to the grade bonus.

Automated tests can fill this gap:
they provide very direct feedback (e.g.\ failing input and expected output pairs) with little delay
without giving away too much information.
Needless to say, they are also an important mechanism
to a large course going at all.
We describe our testing infrastructure in more detail in \cref{sec:tech_setup_test}.

However, we still let tutors manually review all final submissions
in the first iteration of the course.
We specifically instructed them to provide feedback not covered by automation,
in particular regarding code quality.
To our dismay, we have to report that this feedback did very little and
most of it was probably ignored.
We suspect that many students only cared so long as they passed the tests
and never looked at their code and code quality feedback again.

In our second iteration, we hence reallocated ressources:
instead of grading submissions,
tutors now supported us by creating engaging exercises
\kevin{TODO: kevin reference}
and offering new content (e.g.\ running workshops with industrial partners)
while we focused on writing exhaustive tests with good feedback and extended our automated proof checking facilities (see \cref{sec:cyp}).
To provide at least some feedback on
code quality, we instead instructed students
to use a linter (see \cref{sec:tech_setup_test}).
\kevin{TODO: linter gives direct and annoying feedback}

We can report very positively on this reallocation:
we were able to offer a diverser set of exercises and
had the ressources to offer new content
while quality of code did not seem to suffer.

\paragraph{Competition and Awards}
Due to very positive feedback,
we continued the tradition of running an opt-in weekly
programming competition as introduced in \cite{next_1100}.
Each week, one of the homework assignments
was selected as a competition problem
and a criterion for ranking the correct entries was fixed.
The range of problems and ranking critera were very diverse,
including code golf challenges,
optimisation problems,
game strategy competitions,
ACM-like programming contests,
and creative tasks like composition of music
and computer art.
We introduce some of the exercises and share their templates in \cref{sec:selected_exercises}.
The top 30 entries received prestigious points
and were presented to the public on a blog\footnote{\href{https://www21.in.tum.de/teaching/fpv/WS19/wettbewerb.html}{WS19} and \href{https://www21.in.tum.de/teaching/fpv/WS20/wettbewerb.html}{WS20}}
written using ironic self-important third-person style.
The top 30 students then received awards at
a humorously organised award ceremony.

To improve on previous iterations,
we made participation more exciting by organising better awards.
We cooperated with industry partners
to offer prizes such as tickets to functional programming conferences,
Haskell workshops and programming books, as well as cash and material prizes.
This initial contact with industry partners
also sparked the idea to offer real-world Haskell workshops
ran by software engineers from industry in WS20 (explained further below).

The competition in WS20 also greatly benefitted from incooperating the work of our tutors:
At the beginning of the semester,
we brainstormed for competition ideas.
We then formed teams, each one being responsible for the
implemention of one idea to be published as a competition exercise during the semester.
This allowed us to create more extensive, diverse,
and practical exercises than in previous iterations.

As reported in \cite{next_1100},
we can confirm that the competition works extremely well to motivate talented students.
They go very well beyond what is taught as part of the course.
Many of them continue to become major drivers in
the team of tutors in follow-up iterations.
\kevin{TODO: mention that there were more applications}
We also received testimonies from students that even though they did not perform well or did not participate at all in the competition,
they nevertheless enjoyed the humorous blog posts and advanced material discussed on it.
Despite the work of our tutors,
however,
running the competition remains an enormous task,
in particular the evaluation of submissions and
writing of the blog post.
We envisage further assistance by tutors in future iterations in those regards.

\paragraph{Workshops with Industry Partners}
Many students at TUM have questioned the applicability and value of functional programming for real-world applications.
Obviously, there is not much use in us academics telling
them otherwise.
Instead, we had the idea to invite people from industry
to offer special functional programming workshops about
practical topics not covered in our course.
In WS20,
we were able to host three workshops on 1) design patterns
2) networking and advanced IO and 3) user interfaces and multithreading.
We limited participation to 35 students for each workshop,
and to our delight, demand exceeded supply.
Both industry partners and workshop participants
reported very positively to us.
Moreover, organisational overhead was small.
We envisage offering even more workshops in future iterations and highly recommend this mechanisms to demonstrate real-world applicability to other educators.

\paragraph{Social Interactions}
Studies confirmed that the COVID19 severly
students social life,
leading to higher levels of stress, anxiety, loneliness, and depressive symptoms \cite{students_lockdown1}.
This got us thinking about mechanisms we
can employ in WS20 to foster social interaction and exchange between students
-- which play a crucial pedagogical role in general \kevin{TODO: reference}.

Firstly, we decided to employ pair-programming (groups of 3--4 students)
in our online tutorials.
The technical setup for this is desribed in \cref{sec:tech_setup_test}.
This not only made social interaction an integral part of the tutorial,
but also had positive effects on knowledge sharing.
We can report very positively on this policy.

Secondly, we hosted two informal get-together sessions,
one at the beginning and one at the end of the semester.
We started with icebreaker sessions in breakout rooms,
randomly allocating students and at least one tutor in each group.
Following these sessions,
we opened thematic breakoutrooms where students
could freely move and talk about
given topic.
Some students preferred to talk about the course material, others had
light-hearted conversations about university, yet others started to play online games.
All in all, we received very positive feedback for these sessions.

Thirdly, we organised a special
ACM-like programming contest
where students could participate in teams,
followed again by a light-hearted get-together session
for participants.
The contest is described in more detail in \cref{sec:selected_exercises}.
\kevin{TODO: in repo instead}
