\subsection{Selected Exercises}

The design philosophy of each exercise sheet
can be described as follows:
\kevin{TODO: discuss those and change them}
\begin{enumerate}
  \item introduce tasks with incrementing difficulty,
  \item provide instant feedback, and
  \item offer one creative task that can be solved by simple means but also be enhanced by means that go beyond the syllabus.
\end{enumerate}


\paragraph{Overview 2020}
\begin{enumerate}
\item \href{https://www21.in.tum.de/teaching/fpv/WS20/assets/ex01.pdf}{Esparanto}
\item \href{https://www21.in.tum.de/teaching/fpv/WS20/assets/ex02.pdf}{Implementation according to mathematical specification}
\item \href{https://www21.in.tum.de/teaching/fpv/WS20/assets/ex03.pdf}{Sudoku}
\item \href{https://www21.in.tum.de/teaching/fpv/WS20/assets/ex04.pdf}{CYP (structural induction) and parsing code golf}
\item \href{https://www21.in.tum.de/teaching/fpv/WS20/assets/ex05.pdf}{CYP (computation induction + case analysis), type inference}
\item \href{https://www21.in.tum.de/teaching/fpv/WS20/assets/ex06.pdf}{School of Music}
\item \href{https://www21.in.tum.de/teaching/fpv/WS20/assets/ex07.pdf}{Finite Typeclass}
\item \href{https://www21.in.tum.de/teaching/fpv/WS20/assets/ex08.pdf}{Virus Game}
\item \href{https://www21.in.tum.de/teaching/fpv/WS20/assets/ex09.pdf}{CYP structural induction trees}
\item \href{https://www21.in.tum.de/teaching/fpv/WS20/assets/ex10.pdf}{Resolution}
\item \href{https://www21.in.tum.de/teaching/fpv/WS20/assets/ex11.pdf}{Abstract Data Types, Homomorphisms, and Tests}
\item \href{https://www21.in.tum.de/teaching/fpv/WS20/assets/ex12.pdf}{IO, Enumerations, and Programming Contest}
\end{enumerate}

\paragraph{Overview 2019}
\begin{enumerate}
\item Sheet 4: Spellchecking
\item Sheet 10: Domineering
\item Sheet 13: SVG and art
\end{enumerate}
