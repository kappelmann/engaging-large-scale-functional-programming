\subsection{Selected Exercises and Tools}\label{sec:selected_exercises}

\kevin{TODO: mention again that
we focused on having diverse exercises}
In the final version of this article,
this section will introduce a selected list of
useful/creative Haskell exercises and tools
that were created as part of
the course and can be reused by other educators.
All exercises will be made available in a central git repository.
Here are some examples:
\begin{itemize}
\item A mocked IO library that allows property-based IO-testing in Haskell.
\item An implementation of Chain Reaction\footnote{\url{ https://brilliant.org/wiki/chain-reaction-game/}},
including a competition server
and a ranking website with statistics,
interactive, game replay, etc.
\kevin{TODO: link}
\item A music synthesiser framework
\item A turtle graphics\footnote{\url{https://en.wikipedia.org/wiki/Turtle_graphics}} framework
\item A simple framework to run ACM-like programming competitions
\item A framework for UNO\footnote{\url{https://en.wikipedia.org/wiki/Uno_(card_game)}}.
\end{itemize}
Some further examples can be seen on our competition blogs\footnote{\href{https://www21.in.tum.de/teaching/fpv/WS19/wettbewerb.html}{WS19} and \href{https://www21.in.tum.de/teaching/fpv/WS20/wettbewerb.html}{WS20}}.

% The design philosophy of each exercise sheet
% can be described as follows:
% \kevin{TODO: discuss those and change them}
% \begin{enumerate}
  % \item introduce tasks with incrementing difficulty,
  % \item provide instant feedback, and
  % \item offer one creative task that can be solved by simple means but also be enhanced by means that go beyond the syllabus.
% \end{enumerate}


% \paragraph{Overview 2020}
% \begin{enumerate}
% \item \href{https://www21.in.tum.de/teaching/fpv/WS20/assets/ex01.pdf}{Esparanto}
% \item \href{https://www21.in.tum.de/teaching/fpv/WS20/assets/ex02.pdf}{Implementation according to mathematical specification}
% \item \href{https://www21.in.tum.de/teaching/fpv/WS20/assets/ex03.pdf}{Sudoku}
% \item \href{https://www21.in.tum.de/teaching/fpv/WS20/assets/ex04.pdf}{CYP (structural induction) and parsing code golf}
% \item \href{https://www21.in.tum.de/teaching/fpv/WS20/assets/ex05.pdf}{CYP (computation induction + case analysis), type inference}
% \item \href{https://www21.in.tum.de/teaching/fpv/WS20/assets/ex06.pdf}{School of Music}
% \item \href{https://www21.in.tum.de/teaching/fpv/WS20/assets/ex07.pdf}{Finite Typeclass}
% \item \href{https://www21.in.tum.de/teaching/fpv/WS20/assets/ex08.pdf}{Virus Game}
% \item \href{https://www21.in.tum.de/teaching/fpv/WS20/assets/ex09.pdf}{CYP structural induction trees}
% \item \href{https://www21.in.tum.de/teaching/fpv/WS20/assets/ex10.pdf}{Resolution}
% \item \href{https://www21.in.tum.de/teaching/fpv/WS20/assets/ex11.pdf}{Abstract Data Types, Homomorphisms, and Tests}
% \item \href{https://www21.in.tum.de/teaching/fpv/WS20/assets/ex12.pdf}{IO, Enumerations, and Programming Contest}
% \end{enumerate}

% \paragraph{Overview 2019}
% \begin{enumerate}
% \item Sheet 4: Spellchecking
% \item Sheet 10: Domineering
% \item Sheet 13: SVG and art
% \end{enumerate}
