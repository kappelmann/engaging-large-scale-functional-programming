\subsection{Selected Exercises and Tools}\label{sec:selected_exercises}

Many students at TUM have questioned the
applicability and usefulness
of functional languages after completing
their mandatory functional programming course.
We believe this is mainly due to two reasons:
1) introductory programming courses often stick
to simple algorithmic or mathematically inspired challenges and
2) side-effects (in particular IO)
are often introduced very late
in functional programming courses.

Changing the latter appeared unpromising to us:
we think that students would be confused
if a ``special'' IO type and do notation were to
be introduced before they are comfortable
with the basic features of functional
languages.
Thus we tried to crank the other handle
by creating diverse exercises that go beyond
simple terminal applications.
Designing and implementing such exercises,
however, is labour-intensive.
As mentioned in \cref{sec:engagement},
we thus decided to reallocate resources and
let our student assistants help us with this work
rather than providing feedback for homework submissions.

This turned out to be a very fruitful idea:
the quality of our student assistants' work was often way above what
we expected.
The one difficulty we initially faced was the mediocre quality of
tests written by most assistants.
They only had the rudimentary knowledge of QuickCheck taught as part of
the course.
We thus hosted a workshop for our assistants that explained
our testing infrastructure and provided best-practice
patterns when writing tests.
The quality of tests significantly increased following this workshop,
though we still had to polish them before publication.

In the final version of this article,
we will introduce a selected list of
these exercises and some useful tools
that were created as part of the course.
We will also make them available in this article's repository\footnote{\url{https://github.com/kappelmann/engaging-large-scale-functional-programming}}
such they can be reused by other educators.
Here are some examples:
\begin{itemize}
\item A mocked IO library that allows property-based IO-testing in Haskell.
\item An implementation of Chain Reaction\footnote{\url{ https://brilliant.org/wiki/chain-reaction-game/}},
including a competition server
and a ranking website with statistics,
interactive, game replay, etc.
\kevin{TODO: link}
\item A music synthesizer framework
\item A turtle graphics\footnote{\url{https://en.wikipedia.org/wiki/Turtle_graphics}} framework
\item A simple framework to run ACM-like programming competitions
\item A framework for UNO\footnote{\url{https://en.wikipedia.org/wiki/Uno_(card_game)}}.
\end{itemize}
Some further examples can be seen on our competition blogs\footnote{\url{https://www21.in.tum.de/teaching/fpv/WS20/wettbewerb.html} (WS20) and

\url{https://www21.in.tum.de/teaching/fpv/WS19/wettbewerb.html} (WS19)}.

% The design philosophy of each exercise sheet
% can be described as follows:
% \kevin{TODO: discuss those and change them}
% \begin{enumerate}
  % \item introduce tasks with incrementing difficulty,
  % \item provide instant feedback, and
  % \item offer one creative task that can be solved by simple means but also be enhanced by means that go beyond the syllabus.
% \end{enumerate}


% \paragraph{Overview 2020}
% \begin{enumerate}
% \item \href{https://www21.in.tum.de/teaching/fpv/WS20/assets/ex01.pdf}{Esparanto}
% \item \href{https://www21.in.tum.de/teaching/fpv/WS20/assets/ex02.pdf}{Implementation according to mathematical specification}
% \item \href{https://www21.in.tum.de/teaching/fpv/WS20/assets/ex03.pdf}{Sudoku}
% \item \href{https://www21.in.tum.de/teaching/fpv/WS20/assets/ex04.pdf}{CYP (structural induction) and parsing code golf}
% \item \href{https://www21.in.tum.de/teaching/fpv/WS20/assets/ex05.pdf}{CYP (computation induction + case analysis), type inference}
% \item \href{https://www21.in.tum.de/teaching/fpv/WS20/assets/ex06.pdf}{School of Music}
% \item \href{https://www21.in.tum.de/teaching/fpv/WS20/assets/ex07.pdf}{Finite Typeclass}
% \item \href{https://www21.in.tum.de/teaching/fpv/WS20/assets/ex08.pdf}{Virus Game}
% \item \href{https://www21.in.tum.de/teaching/fpv/WS20/assets/ex09.pdf}{CYP structural induction trees}
% \item \href{https://www21.in.tum.de/teaching/fpv/WS20/assets/ex10.pdf}{Resolution}
% \item \href{https://www21.in.tum.de/teaching/fpv/WS20/assets/ex11.pdf}{Abstract Data Types, Homomorphisms, and Tests}
% \item \href{https://www21.in.tum.de/teaching/fpv/WS20/assets/ex12.pdf}{IO, Enumerations, and Programming Contest}
% \end{enumerate}

% \paragraph{Overview 2019}
% \begin{enumerate}
% \item Sheet 4: Spellchecking
% \item Sheet 10: Domineering
% \item Sheet 13: SVG and art
% \end{enumerate}
