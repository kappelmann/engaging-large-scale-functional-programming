\section{Conclusion}\label{sec:conclusion}

As computer science departments continue to grow,
COVID-19 continues to spread,
and imperative programming appears to be the industry norm,
we functional programming educators
have to find ways to make our courses more scalable and
engaging, while demonstrating the elegance and usefulness of functional programming and having to adapt to a mix of teaching in physical and virtual space.
We hope our efforts not only inspired our 2000 students
but also other educators to take on these challenges.
The insights and resources presented in this article proved valuable to us and will hopefully also do to others.

For future iterations,
we plan to keep many things that we introduced during the pandemic and double down on some new engagement mechanisms such as the supplementary workshops.
One thing we want to improve is the
time we have to spend on the weekly competition,
for example by handing some of the workload to student assistants and creating fewer but more engaging exercises.
Another possibility we want to explore is to use more
competition exercises that can be automatically graded
and their results continuously be published on a website (like our Chain Reaction competition in \cref{sec:selected_exercises}).
This would also increase student engagement due to instant feedback.
We also plan to stick to the digital open-book exam format,
though as of yet, we have no good solution to prevent cheating.

\paragraph{Acknowledgements}
We want to thank all people involved in the course,
in particular Tobias Nipkow,
Manuel Eberl,
our student assistants,
the ArTEMiS development team,
our industry partners
Active Group,
QAware,
TNG Technology Consulting,
and Well-Typed,
and our 2000 Haskell students.
Finally, we thank the participants of TFPIE 2022
and the anonymous reviewers for their valuable feedback.
