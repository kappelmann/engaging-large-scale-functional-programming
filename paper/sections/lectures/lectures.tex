\section{Lectures}\label{sec:lectures}

The lectures used a mix of
slides,
live coding, and whiteboard proofs.
Each self-contained topic comes with small case studies and examples.
The latter are accompanied by suitable QuickCheck tests
and inductive correctness proofs when appropriate.
The proofs presented during the lectures stay close to the format accepted by the proof checker that is used in the practical part of the course (see~\cref{sec:cyp}).
The live coding sections in particular received positive feedback by the students.

In both iterations, students were allowed to interact and ask questions at any time.
However, synchronous interaction with hundreds of students is challenging:
While the lecturer cannot answer all questions due to time constraints,
many students are also too reserved to ask questions given the large audience.
As the semester progresses, interaction tends to degrade to
questions posed by a small community of motivated students;
questions shared by a majority, on the other hand, often go unheard.

In WS19, we partly addressed this problem by offering an asynchronous Q\&A forum\footnote{We used a Zulip instance hosted by our department: \url{https://zulip.com/case-studies/tum/}}
where students could post anonymously or using their real name.
The forum contained separate sections for the theoretical part (including the lectures)
and the practical part of the course.
Questions posted in the former were answered by the lecturer
to increase interaction between students and the lecturer -- the lack of which was often criticized by students in our department.
Questions in the latter were also answered by teaching assistants and by other students.
Answers by students were explicitly encouraged by us and
outstanding contributions were even awarded a special prize at the end of the semester.

However, while the forum was a great success with over 3800 posts per semester,
engagement in the theoretical section stayed far behind its practical counterpart ($\leq$ 2\%).
Moreover, the forum does not fully address the live interaction problem since
questions are answered asynchronously.
As such, students stuck with conceptional problems may not be able
to keep up with the rest of the lecture,
leading to frustration.

For the second iteration of the course -- the online semester --
we thus added a new interaction method.
The lectures were livestreamed and
interaction was made possible by means of a live Q\&A board\footnote{We used tweedback \url{https://tweedback.de/}}.
The board was moderated by a PhD student
sitting in the same room as the lecturer.
Questions could be answered and voted on by students as well as the moderator.
The moderator approved and answered individual and simple questions directly while forwarding questions of general interest to the lecturer, in order to increase engagement between the students and the lecturer.

We can report that this moderated format increased engagement when compared to the first iteration:
students were less reluctant to submit questions because
1) they had the chance to ask questions anonymously,
2) they were not afraid to ``interrupt'' the lecture, and
3) they were able to ask new kinds of questions.
Examples of the third include discussion of alternative solutions by students,
organisational questions,
and slightly off-topic discussions that nevertheless increase engagement and curiosity.
We recommend to offer a moderated live Q\&A board even for lectures taking place on campus or running in a hybrid format.

Following the livestream,
the recordings were uploaded for asynchronous consumption.
Students watching the lectures asynchronously still had the chance to submit questions to the forum and to receive answers by the lecturer.

