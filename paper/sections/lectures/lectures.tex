\section{Lectures}

Concepts are introduced in self-contained small steps, one by one.
Characteristic features of functional programming languages such as
higher-order functions and algebraic data types are
only introduced midway through the course.
This makes the design of interesting practical tasks harder
but ensures that students are not overwhelmed by the diversity
of new principles that are not part of introductory imperative programming courses.

Each self-contained topic comes with small case studies and examples.
The latter are accompanied by suitable QuickCheck tests
and inductive correctness proofs when appropriate (more details in \cref{sec:cyp}).

The lecturer presents topics using a mix of
slide presentations,
live coding, and whiteboard proofs.
Students can interact and ask questions at any time.
During the online semester,
lectures were livestreamed and
engagement made possible by means of a live Q\&A stream.
The stream was moderated by a PhD student
sitting in the same room as the lecturer.
Questions could be answered by students as well as the moderator.
The moderator approved and answered individual and simple questions directly while putting forward those questions to the lecturer
that were of a more general nature.
Those questions were not put forward due to a lack of understanding of the moderator
but to increase engagement between students and the lecturer,
a point of critic often raised by students in our department.

We can report that this moderated format increased engagement when compared to the physical format of the first semester:
students were less reluctant to submit questions for
1) they had the chance to ask question anonymously,
2) they were not afraid to ``interrupt'' the lecture, and
3) they were able to ask new kind of questions.
Examples of the third include discussion of alternative solutions by students,
organisational questions,
and slightly off-topic discussions that nevertheless increase engagement and curiousity.

Following the livestream,
the recordings were also uploaded for asynchronous consumption.

