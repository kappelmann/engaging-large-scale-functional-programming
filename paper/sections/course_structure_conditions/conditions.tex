\subsection{Conditions}

The 5~ECTS\footnote{European Credit Transfer System; one ECTS credit equals 30 hours of work} course was mandatory for computer science undergraduates in their third semester and
an elective for other related degrees such as games engineering or information systems.
All students studied Java in their first semester and had taken courses on algorithms and data structures,
discrete mathematics, and linear algebra.
The course ran for 14 weeks with
one 90-minute lecture,
one 90-minute tutorial,
and one exercise sheet each week.

1057 students registered for
the first iteration\footnote{\url{https://www21.in.tum.de/teaching/fpv/WS19/} (website -- except ``Wettbewerb'' -- German; course material English)} that took place on campus in winter semester 2019 (WS19) and
1031 for the second iteration\footnote{\url{https://www21.in.tum.de/teaching/fpv/WS20/} (English)} in winter semester 2020 (WS20), taking place in virtual space due to the COVID-19 pandemic.

Both iterations were organised by the lecturer, Tobias Nipkow, and the authors of this paper.
The former designed the course, created the slides\footnote{\url{https://www21.in.tum.de/teaching/fpv/WS20/assets/slides.pdf}}, and delivered the lectures.
The others took care of the practical and organisational part of the course.
All gained valuable experience in running an online course on the theory of computation for 1071
students in summer semester 2020.
Finally, Manuel Eberl had the honour of assisting us with the weekly programming competition in his role as the \emph{Master of Competition} (see \cref{sec:engagement}).

Needless to say,
running tutorials for more than 1000
students each semester on our own is impossible.
We were further assisted by
13 student assistants in WS19 and
22 student assistants in WS20.
In WS19, their primary job was to run the tutorials and provide feedback for homework submissions (e.g.\ code quality).
However, it became clear to us
that this manual feedback is not effective
and that their time is better spent creating engaging exercises (see \cref{sec:engagement}).

% As written before, the rapid growth in number of students (225\% in seven years)
% far exceeds the growth in academic staff (121\%).


