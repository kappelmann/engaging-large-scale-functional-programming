\section{Check Your Proof by Example}\label{sec:cyp}
Besides functional programming,
the course also dealt with the verification of functional programs.
Even though we only consider a strict and total subset
of Haskell for our proofs,
equational reasoning together with induction
(and case analysis)
is already sufficient to prove interesting properties.
Since ``fast and loose reasoning is morally correct''~\cite{fast_and_loose},
valid properties in this sub-language carry over to Haskell.
Simple inductive proofs lend themselves well to be automatically checked and, as announced in~\cite{next_1100}, a tool called ``Check Your Proof'' (CYP for short) was developed at our lab by Durner and Noschinski\footnote{\url{https://github.com/noschinl/cyp}}.

The first example of such a proof presented in the lecture is the proof of associativity of the append-operator for lists.
Proving this example in CYP first requires us
to define the data type of lists.
% We use
% the constructors \lstinline[style=cyp]![]! and \lstinline[style=cyp]!:!
% as is standard in Haskell.
\begin{lstlisting}[style=cyp]
  data List a = [] | a : List a
\end{lstlisting}
Now we can define the infix append-operator \lstinline[style=cyp]!++!
\begin{lstlisting}[style=cyp]
  [] ++ ys = ys
  (x : xs) ++ ys = x : (xs ++ ys)
\end{lstlisting}
and state the goal
\begin{lstlisting}[style=cyp]
  goal xs ++ (ys ++ zs) .=. (xs ++ ys) ++ zs
\end{lstlisting}
where all variables are implicitly universally quantified.
The listings above describe the background theory of the proof.
The theory is fixed in a file and provided
to the students.
The students then supply the proof in a separate file.
In our case, we proceed to prove the statement by structural induction on \lstinline[style=cyp]!xs!.
\begin{lstlisting}[style=cyp]
  Lemma: xs ++ (ys ++ zs) .=. (xs ++ ys) ++ zs
  Proof by induction on List xs
    Case []
      To show: [] ++ (ys ++ zs) .=. ([] ++ ys) ++ zs
      Proof
                        [] ++ (ys ++ zs)
        (by def ++) .=. ys ++ zs
        (by def ++) .=. ([] ++ ys) ++ zs
      QED
\end{lstlisting}

In above listing, the first \lstinline[style=cyp]!Proof! marks the beginning of an inductive proof whereas the second \lstinline[style=cyp]!Proof!, which has no further arguments, starts an equational proof.
While CYP allows to arbitrarily nest proofs
by case analysis or induction,
the innermost goal must always be discharged by an equational proof.
An equational proof is a chain of equations that rewrite the left-hand side of the current goal to the right-hand side.
The user has to justify each step by a corresponding equation that yields the right-hand side
when applied to the term or a subterm on the left-hand side.
Note that in the example, \lstinline[style=cyp]!(by def ++)! refers to either one of the defining equations of \lstinline[style=cyp]!++!
and CYP will check if any of them justifies the current step.

Now, consider the inductive case.
\begin{lstlisting}[style=cyp]
    Case x : xs
      To show: (x : xs) ++ (ys ++ zs) .=. ((x : xs) ++ ys) ++ zs
      IH: xs ++ (ys ++ zs) .=. (xs ++ ys) ++ zs
      Proof
                          (x : xs) ++ (ys ++ zs)
        (by def ++)   .=. x : (xs ++ (ys ++ zs))
        (by IH)       .=. x : ((xs ++ ys) ++ zs)

                          ((x : xs) ++ ys) ++ zs
        (by def ++)   .=. (x : (xs ++ ys)) ++ zs
        (by def ++)   .=. x : ((xs ++ ys) ++ zs)
      QED
  QED
\end{lstlisting}
Again, CYP requires the user to be explicit: the goal in each inductive case and the induction hypotheses have to be spelled out.
Note that CYP also allows one to start rewriting from the left-hand side as well as the right-hand side of the goal.

The lecture goes beyond proofs by structural induction and introduces proofs by extensionality, case analysis, and computation induction,
all of which CYP supports with some conditions applying as we will describe shortly.
Computation induction was not supported in the original version by Durner and Noschinski but was only introduced in a fork by us in WS20\footnote{\url{https://github.com/lukasstevens/cyp}}.
CYP also allows proving named auxiliary lemmas,
which is useful to modularise proofs
and often necessary to prove generalisations of a given goal.

The simplicity of CYP both in its usage and its implementation comes with some caveats:
\begin{itemize}
    \item The version of CYP used by us is untyped and thus unsound if the background theory contains multiple types as demonstrated in~\cite{cyp_holes}: given a singleton type \lstinline[style=cyp]!data U = U! in the background theory,
      one can prove \lstinline[style=cyp]!x .=. y! by case analysis.
        In an untyped environment, one can then use this lemma to prove equality between any two terms.
    \item Barring soundness issues due to a lack of types, one also needs to ensure that all function definitions are total and that their patterns do not overlap.
    \item Computation induction is unsound if there are recursive calls in branches of an if-then-else expression.
      This is because the induction hypotheses would have to be conditional rewrite rules in such cases, which CYP does not support.
\end{itemize}

Renz et al.~\cite{cyp_holes} discuss the inner workings of CYP in detail and develop an extension that introduces types,
thus solving the first issue.
They also made it possible to leave holes in proofs and in expressions, which then have to be filled in by students.
Solving the other issues, however, would incur additional effort.
We think that CYP should be put on a stronger foundation,
such as higher-order logic (HOL),
without compromising its simplicity from a user perspective.
One possibility would be to rewrite CYP as a frontend to Isabelle/HOL~\cite{isabelle}.
Since HOL is a typed logic, it would solve the first issue.
The latter issues could be resolved using Isabelle's function package~\cite{isabelle_functions}.
A stronger foundation would also allow one to develop new extensions for CYP more confidently.

Our teaching experience with CYP has been very positive.
CYP proof checking is quick and can easily be integrated in the testing framework (Tasty) that we used for our programming exercises.
This allowed us to deal with programming and proof exercises in a uniform and scalable way.
CYP was also generally liked by our students:
% as their feedback testifies:
In WS19,
we received two positive comments and one negative comment without asking for feedback on CYP.
When asked explicitly about their thoughts on CYP in WS20,
the students answered with 18 positive comments and three negative comments.
The students liked the instantaneous feedback that CYP provides,
which helped them to deepen their understanding of inductive proofs at their own pace.
The main criticism of CYP was the lack of documentation.
A CYP-cheatsheet was later added to the repository of CYP to improve the situation.

In our exams, two exercises (out of $\approx 8$) were concerned with inductive proofs.
The first exercise was a straightforward proof by structural induction while the second one required more effort, e.g.\ a generalisation of the goal or a slightly less trivial computation induction.
We did not require the students to stick to CYP's syntax in the exam, but we urged them to follow a similar structure,
which worked very well overall and simplified the grading.
However, we have no data on whether CYP also improves students' understanding of induction.

