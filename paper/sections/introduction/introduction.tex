\section{Introduction}

This paper reports on strategies and solutions employed to
run two iterations of a large-scale functional programming and verification course at the Technical University of Munich (TUM).
While the first iteration (winter semester 2019, 1057 participants)
took place on campus,
the second iteration (winter semester 2020, 1031 participants) was affected by the COVID-19 pandemic and took place in virtual space.
Previous iterations of the course were introduced in~\cite{next_1100};
however, we were facing two novel challenges:

\paragraph{Soaring Enrollments}
The relatively young field of computer science has
become one of the largest study programmes around the globe.
The increase of student enrollments is dramatic
\cite{comp_sci_growth1,comp_sci_growth2}
while employment of new teaching staff often lags behind.
At TUM, the number of new enrollments in computer science more than doubled between 2013 and 2020 from 1110 to 2508 (an increase of 125\%)
, while academic staff only increased from 435 to 529 (21\%) \cite{tum_numbers}.

This drastic increase not only requires more physical resources -- like larger lecture halls and more library spaces --
but also academic staff for supervision.
Given the discrepancy in growth between student enrollments and staff employment,
automation of supervision and feedback mechanisms is inevitable.
Automation, however, should not
negatively affect the quality of the teaching.

\paragraph{Online Teaching}
The ongoing COVID-19 pandemic forced a radical
transition from on-site teaching to online classes.
Lecturers had to rethink the way they present material and interact with students,
teaching assistants the way they assist students in tutorial sessions.
Students, on the other hand, suffer from a lack of social interaction and communication, leading to higher
levels of stress, anxiety, loneliness, and symptoms of depression \cite{students_lockdown1}.

In our experience, the disconnect between students and lecturers, as well as the lack of on-campus interaction between students may also lead to \emph{cramming}:
the practice of showing little participation during the semester
while studying extensively just before the exam.
Cramming tends to result in poor long-term retention and shallow understanding of material.
Indeed, the benefit of spacing learning events apart rather than cramming has been demonstrated in hundreds of experiments \cite{cramming1,cramming2}.

\vspace{\baselineskip}\noindent
Besides these two general challenges,
there is a third -- subject-specific --
challenge we were keen to tackle:

\paragraph{Functional Programming is Practical \kevin{TODO: better paragraph title}}
Feedback by students and personal experience has shown us that many students
at TUM question the applicability and usefulness
of functional languages beyond
the academia.
They are disappointed by a lack of industrial insight
and real world -- or at least interactive -- applications.
Indeed, some even perceive functional programming as an obstacle;
after all, they already know how to program imperatively.

Good educators do not just teach but inspire:
we have to bring the benefits of functional languages
closer to our students' hearts
by showing real-world applicability and making functional programming fun and engaging.


\vspace{\baselineskip}\noindent
In this paper,
we present our answers to these challenges
and provide tools for other educators
to make functional programming lectures -- in physical or virtual space -- engaging and scalable.

\paragraph{Outline}

We begin by describing the general framework and underlying conditions of the course and its syllabus in \cref{sec:course_structure_conditions}.
In \cref{sec:lectures},
we describe the tools and teaching methods that we used during lectures.
\cref{sec:practical_part} describes the
mechanisms and technical setup
we used to create an engaging experience
for the practical part of the course,
including tutorials, homework assignments,
competition systems, and more.
\cref{sec:cyp} introduces
``Check Your Proof'' -- a tool created
by our lab that automatically checks simple inductive proofs for Haskell programs.
\cref{sec:exam} describes how we adapted our exams to the COVID-19 situation and the large number of participants,
and \cref{sec:conclusion} concludes with a summary and aspects to improve in future.

