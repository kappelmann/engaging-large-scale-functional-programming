\section{Introduction}

This paper reports on strategies and solutions employed to
run two iterations of a large-scale functional programming and verification course at the Technical University of Munich (TUM).
While the first iteration (winter semester 2019/20, TODO participants)
took place on campus,
the second iteration (winter semester 2020/21, TODO participants) was affected by the COVID-19 pandemic and took place in virtual space.
A similar course was offered before \citep{next_1100};
however, we were facing two novel and severe challenges:

\paragraph{Soaring Enrollments}
The relatively young field of computer science has
become one of the largest study programmes around the globe.
The increase of student enrollments is dramatic
\citep{comp_sci_growth1,comp_sci_growth2}
while employment of new teaching staff often lacks behind.
At TUM, the number of new enrollments in computer science between 2013 and 2020 alone
more than doubled from 1110 to 2508 (225\%)
while academic staff increased from 435 to 529 (121\%) \citep{tum_numbers}.

This drastic increase not only asks for more physical ressources -- like larger lecture halls and more library spaces --
but also academic staff for supervision.
Given the discrepancy in growth of student enrollments and staff employment,
automation of supervision mechanisms renders inevitable.
Automation, however, should not
negatively effect the quality of supervision.

\paragraph{Online Teaching}
The ongoing COVID-19 pandemic forced a radical
transition from on-site teaching to online classes.
Lecturers have to rethink the way they present material and interact with students,
teaching assistants the way they assist students in tutorial sessions.
Students, on the other side, suffer from lack of social interaction and communication.
\kevin{TODO: write more}

All this might not only lead to negative psychological consequences but also \emph{cramming}:
the practice of showing little participation during the semester
while studying extensively just before the exam.
Cramming tends to result in poor long-term retention and shallow understanding of material.
Indeed, the benefit of spacing learning events apart rather than cramming has been demonstrated in hundreds of experiments \citep{cramming1,cramming2}.

\vspace{\baselineskip}\noindent
Besides these two general challenges,
there is a third subject-specific
question we were keen to answer:

\paragraph{Why Bother About Functional Programming?}
Feedback has shown that many students
at TUM question the applicability and usefulness
of functional languages beyond
the academic world of universities.
They are disappointed by a lack of industrial insight
and real world applications.
Indeed, some even perceive it as an obstacle;
after all, they already know how to program imperatively.

Good educators do not just teach but inspire:
we have to bring the benefits of functional languages
closer to our students' hearts
while showing real-world applicability and making functional programming fun and engaging.


\vspace{\baselineskip}\noindent
In this paper,
we describe our answers to these challenges
and provide tools for other educators
to make functional programming lectures engaging,
large-scale, and work in both physical and virtual space.

\paragraph{Outline}
\begin{enumerate}
\item automate feedback
\item Foster engagement to avoid cramming
\item Offer chances to go beyond what is taught
\item Support social interaction
\item Provide good and uniform tooling
\end{enumerate}

To prevent cramming,
we used a variety of engagement mechanisms and incentives as the course progresses.
Good engagement mechanisms not only keep students busy
but genuinely make the course more fun.
Participation should originate from curiosity and enjoyment rather than pressure.
In the best case, students might even study beyond the material presented
if given the possibility.

