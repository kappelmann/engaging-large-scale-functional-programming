\section{Related Work}\label{sec:related_work}

% TODO create two subparagraphs:
% one about engagement mechanisms in higher education and
% one about tools/mechanisms to run large-scale, (functional)
% programming classes.

The literature on student engagement in higher-education is vast,
and we are not in the position to give a thorough account of it.
Some general accounts of and suggestions for
student engagement can be found in~\cite{student_engagement,engagementproposals}
and investigations focusing on online courses
in~\cite{onlineengagement3,onlineengagement2,onlineengagement4,onlineengagement5,onlineengagement1}.
Many of the mechanisms described there were employed in our course.
As such, our work can be seen as a case study, testing the hypotheses made in said literature.

Much work has been spent on the
automated assessment of programming exercises.
A recent overview can be found in~\cite{automatedassessment}.
We do not aim to provide another such software package but
to describe how various existing solutions
can be put together to
address the challenges laid out in~\cref{sec:intro}.

Various articles have been published about
specific aspects and tools when teaching functional programming,
including work on gamification~\cite{soccerfun},
IO testing~\cite{iotest2}, and
visualisation of program evaluation~\cite{steppingocaml}.
While our work contributes to this realm,
it also describes more general strategies
to run engaging (functional) programming courses.

Some case studies of engagement strategies in programming courses are given in~\cite{next_1100,engagingprogramming,largeprogrammingclass}.
In particular,~\cite{next_1100}
introduces a previous iterations of the course,
focusing on the lecture material,
specific exercises, and a custom-made testing infrastructure.
Our main objective, instead, is to provide strategies and resources to address
the challenges laid out in~\cref{sec:intro},
i.e.\ the drastic increase in student enrolments,
the challenges of online teaching,
and the difficulty of demonstrating the practicality of functional programming.
To the best of our knowledge,
there are only few papers about converting physical computer science classes to virtual ones
and none that deal with all given challenges.
% We also give an account of the improved testing infrastructure
% that supersedes the one introdued in~\cite{next_1100}.

