\section{Related Work}\label{sec:related_work}

The literature on student engagement in higher-education is vast,
and we are not in the position to give a thorough account of it.
Some general accounts of and suggestions for
student engagement can be found in~\cite{student_engagement,engagementproposals}
while work specifically focusing on online courses can be found in~\cite{onlineengagement3,onlineengagement2,onlineengagement4,onlineengagement5,onlineengagement1}.
Many of the mechanisms described there were employed in our course.
As such, our work can be seen as a case study, testing the hypotheses framed in these works.

Much work has been spent on the
automated assessment of programming exercises.
A recent overview can be found in~\cite{automatedassessment}.
We do not to provide another such software package but
describe how various existing solutions
can be put together to
address the challenges laid out in~\cref{sec:intro}.

To the best of our knowledge,
there are only few papers about converting physical computer science classes to virtual ones
and none that deal with all the challenges outlined by us.
Two experience reports using imperative languages can be found in~\cite{largeprogrammingclass,onlinecourse1}.
While they use a project-based learning approach,
that is students work in teams on graded projects,
our students had to submit their homework individually.
We fostered social interaction and team-based learning
in the context of the tutorials, workshops,
and programming contests instead.
We believe that this strikes a good balance
between individual work and team work.

There are several studies and experience reports that corroborate some of our findings, for example
\begin{itemize}
  \item increasing engagement using live programming~\cite{fp_first_year_risks_benefits,livecoding1,livecoding2}, pair-programming~\cite{engagingprogramming,teaching_fp_first_year}, and gamification~\cite{soccerfun,teaching_fp_glossy_games,teaching_fp_macedonian},
  \item letting students ask questions during the lecture via chat~\cite{increase_interest_fp},
  \item using automated grading for prompt feedback~\cite{teaching_fp_glossy_games,teaching_art_fp_automated,teaching_fp_macedonian}, and
  \item introducing I/O early (without explaining monads)~\cite{fp_first_year_risks_benefits,haskell_school_hudak,teaching_fp_chalmers} and establishing connections to industry by organising events~\cite{teaching_fp_chalmers} to demonstrate ``real-world'' applicability.
\end{itemize}

Various articles have been published about
specific tools to teach functional programming,
such as I/O testing~\cite{iotest2}
and visualisation of program evaluation~\cite{steppingocaml}.
While our work contributes to this realm,
it also describes more general strategies
to run engaging (functional) programming courses.

Finally,~\cite{next_1100}
introduces a previous iteration of our course,
focusing on the lecture material,
specific exercises, and a custom-made testing infrastructure.
Our objective, instead, is to provide strategies and resources to address
the challenges laid out in~\cref{sec:intro},
i.e.\ the drastic increase in student enrolments,
the challenges of online teaching,
and the difficulty of showing the practicality of functional programming.
