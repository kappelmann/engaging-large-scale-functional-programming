\section{Related Work}\label{sec:related_work}

The literature on student engagement in higher-education is vast,
and we are not in the position to give a thorough account of it.
Some general accounts of and suggestions for
student engagement can be found in~\cite{student_engagement,engagementproposals}
while work specifically focusing on online courses can be found in~\cite{onlineengagement3,onlineengagement2,onlineengagement4,onlineengagement5,onlineengagement1}.
Many of the mechanisms described there were employed in our course.
As such, our work can be seen as a case study, testing the hypotheses framed in these works.

Much work has been spent on the
automated assessment of programming exercises.
A recent overview can be found in~\cite{automatedassessment}.
We do not to provide another such software package but
describe how various existing solutions
can be put together to
address the challenges laid out in~\cref{sec:intro}.

Various articles have been published about
specific aspects and useful tools when teaching functional programming,
including work on gamification~\cite{soccerfun},
I/O testing~\cite{iotest2}, and
visualisation of program evaluation~\cite{steppingocaml}.
While our work contributes to this realm,
it also describes more general strategies
to run engaging (functional) programming courses.

Some case studies of engaging programming courses are given in~\cite{next_1100,engagingprogramming,largeprogrammingclass}.
In particular,~\cite{next_1100}
introduces a previous iteration of our course,
focusing on the lecture material,
specific exercises, and a custom-made testing infrastructure.
Our objective, instead, is to provide strategies and resources to address
the challenges laid out in~\cref{sec:intro},
i.e.\ the drastic increase in student enrolments,
the challenges of online teaching,
and the difficulty of demonstrating the practicality of functional programming.
To the best of our knowledge,
there are only few papers about converting physical computer science classes to virtual ones
and none that deal with all the challenges outlined by us.
% We also give an account of the improved testing infrastructure
% that supersedes the one introdued in~\cite{next_1100}.

Some of our findings are corroborated by previous experience reports, e.g.
\begin{itemize}
  \item increasing engagement using live programming~\cite{fp_first_year_risks_benefits}, pair-programming~\cite{teaching_fp_first_year}, and gamification~\cite{soccerfun,teaching_fp_glossy_games,teaching_fp_macedonian},
  \item introducing IO early without explaining monads~\cite{fp_first_year_risks_benefits,teaching_fp_chalmers} to demonstrate that functional is no less powerful than imperative programming as Hudak does in their textbook~\cite{haskell_school_hudak},
  \item establishing connections to industry by organising events~\cite{teaching_fp_chalmers},
  \item using automated grading for instantaneous feedback~\cite{teaching_art_fp_automated,teaching_fp_macedonian,teaching_fp_glossy_games}, and
  \item letting students ask questions during the lecture via chat~\cite{increase_interest_fp}.
   
\end{itemize}

